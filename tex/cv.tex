% LaTeX Curriculum Vitae Template
%
% Copyright (C) 2004-2009 Jason Blevins <jrblevin@sdf.lonestar.org>
% http://jblevins.org/projects/cv-template/
%
% You may use use this document as a template to create your own CV
% and you may redistribute the source code freely. No attribution is
% required in any resulting documents. I do ask that you please leave
% this notice and the above URL in the source code if you choose to
% redistribute this file.

\documentclass[letterpaper]{article}

\usepackage{hyperref}
\usepackage{geometry}

% Comment the following lines to use the default Computer Modern font
% instead of the Palatino font provided by the mathpazo package.
% Remove the 'osf' bit if you don't like the old style figures.
\usepackage[T1]{fontenc}
\usepackage[sc,osf]{mathpazo}

% Set your name here
\def\name{Roman Perehonchuk}

% Replace this with a link to your CV if you like, or set it empty
% (as in \def\footerlink{}) to remove the link in the footer:
\def\footerlink{http://jblevins.org/projects/cv-template/}

% The following metadata will show up in the PDF properties
\hypersetup{
  colorlinks = true,
  urlcolor = black,
  pdfauthor = {\name},
  pdfkeywords = {economics, statistics, mathematics},
  pdftitle = {\name: Curriculum Vitae},
  pdfsubject = {Curriculum Vitae},
  pdfpagemode = UseNone
}

\geometry{
  body={6.5in, 8.5in},
  left=1.0in,
  top=1.25in
}

% Customize page headers
\pagestyle{myheadings}
\markright{\name}
\thispagestyle{empty}

% Custom section fonts
\usepackage{sectsty}
\sectionfont{\rmfamily\mdseries\Large}
\subsectionfont{\rmfamily\mdseries\itshape\large}

% Other possible font commands include:
% \ttfamily for teletype,
% \sffamily for sans serif,
% \bfseries for bold,
% \scshape for small caps,
% \normalsize, \large, \Large, \LARGE sizes.

% Don't indent paragraphs.
\setlength\parindent{0em}

% Make lists without bullets
\renewenvironment{itemize}{
  \begin{list}{}{
    \setlength{\leftmargin}{1.5em}
  }
}{
  \end{list}
}

\begin{document}

% Place name at left
{\huge \name}

% Alternatively, print name centered and bold:
%\centerline{\huge \bf \name}

\vspace{0.25in}

\begin{minipage}{0.45\linewidth}
  \begin{tabular}{ll}
    Phone: & 38 (063) 756-65-41 \\
    Email: & \href{mailto:roman.peregonchuk@gmail.com}{\tt roman.peregonchuk@gmail.com} \\
    Homepage: & \href{http://perehonchuk.com}{\tt http://perehonchuk.com} \\
  \end{tabular}
\end{minipage}


\section*{Personal}

\begin{itemize}
\item Born on September 29, 1895.
\item United States Citizen.
\end{itemize}


\section*{Education}

\begin{itemize}
  \item B.S. Journalism, Washington University, 1919.

  \item M.A. Mathematics, Washington University, 1921.

  \item Ph.D. Mathematics, Princeton University, 1924.
\end{itemize}


\section*{Employment}

\begin{itemize}
\item Stanford University 1927--1931.
\item Columbia University 1931--1946.
\item University of North Carolina, 1946--1973.
\end{itemize}


\section*{Publications}

\subsection*{Journal Articles}

\begin{itemize}
\item A General Mathematical Theory of Depreciation, 1929, {\it Journal
    of The American Statistical Association} 20, 340--353.
\item Differential Equations Subject to Error, 1927, {\it Journal of The
    American Statistical Association}.
\item Applications of the Theory of Error to the Interpretation of
  Trends (with H. Working), 1929, {\it Journal of the American
    Statistical Association}.
\end{itemize}

\subsection*{Proceedings}

\begin{itemize}
\item A generalized T-Test and measure of multivariate dispersion,
  Proc. Second Berkeley Symposium of Mathematical Statistics and
  Probability, 1951.
\end{itemize}

\bigskip

% Footer
\begin{center}
  \begin{footnotesize}
    Last updated: \today \\
    \href{\footerlink}{\texttt{\footerlink}}
  \end{footnotesize}
\end{center}

\end{document}
